\begin{abstract}
Today big data analytics frameworks running on managed runtimes, such as Java
Virtual Machines (JVM) are extensively used in data centers for analyzing large
datasets. Existing research has explored extending the managed heap using fast
storage devices (e.g., NVMe SSDs) and remote memory. However, a comprehensive
comparison of these techniques is lacking, underscoring the need for detailed
studies evaluating their performance, and reliability.

In our comparative analysis, we investigate strategies for optimizing block
storage access and sizing, focusing on local NVMe SSDs versus remote storage
devices and DRAM. We break down and examine the latencies of local and remote
storage systems, including SPDK NVMe over Fabrics (NVMe-oF) and Network Block
Device (NBD). Using TeraHeap, which extends JVM capabilities by utilizing a
second heap on a storage device, we compare its performance with both local and
remote NVMe storage devices. Our evaluation, conducted using 13 widely-used
applications in two real-world big data frameworks, Spark and Giraph,
demonstrates that remote NVMe can match the performance of local NVMe.

\end{abstract}

