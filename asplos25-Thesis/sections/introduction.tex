\section{Introduction}
%% Demand for more memory
Big data analytics frameworks running on managed runtimes, such as Java virtual
machines (JVM) are widely deployed in data-centers to perform data analysis over
large amount of datasets. The amount of data increases with high rate but DRAM
capacity in a single server scales slower than data growth. Existing approaches
study the extension of the managed heap over fast storage devices (e.g., NVMe
SSDs) and remote memory.

%Today, cloud infrastructures running data-intensive applications encounter
%numerous challenges in transferring data to-from block storage devices. 
%With the
%escalating demands of big data frameworks, the insufficiency of DRAM in scaling
%to such magnitudes becomes apparent. Additionally, traditional approaches to
%accessing local storage devices struggle to match the pace of data processing,
%thereby becoming an overhead. Consequently, the rapid expansion of data
%necessitates a swift and efficient mechanism for data movement to storage
%devices.
\par Extended heap approaches can be categorized based on their implementation over
different storage types: local storage devices, remote storage devices, remote memory. Some systems use the capacity of a fast local storage device (e.g TeraHeap) by an extended heap over NVMe. In contrast, other works use remote storage device
implementations. Other systems see that utilizing remote DRAM for
extended heap operations (e.g Semeru, MemLiner, and Mako) offers high-speed
access and minimize latency and ensure fast data transfer. However, current research lacks a comprehensive comparison of these
techniques. None of the existing work adequately addresses the pros and cons of
each approach or explains why one might be preferable over the others. This gap
highlights a need for detailed evaluative studies that assess the performance,
and reliability of each heap extension implementation
method.

%categorization of how you can have an extended heap approaches like using a extended heap over local storage device or over remote storage device or remote DRAM memory exist...

%However, non of existing work provide a comparison between the techniques on what are the pros and cons and why they are better from one another...


\par In this \textbf{comparative analysis}, we delve into possible strategies for
achieving faster and more efficient block storage accessing while ensuring an
appropriate size of the block storage. The first aspect of our investigation
centers on comparing local storage devices, exemplified by NVMe SSDs, against
remote storage devices and remote DRAM memory. NVMe SSDs, offer rapid data
access and reliability. However, the efficiency of local storage devices in
comparison to remote options remains a question mark and forms the core of our
study. Remote DRAM memory, despite its remote positioning, may offer lower
latency compared to remote storage devices due to its faster access times.
However, its capacity is typically more constrained than storage devices. It's
crucial to note the role of technological advancements such as SPDK NVMe over
Fabrics (NVMe-oF) in mitigating latency concerns. SPDK NVMe-oF holds promise in
reducing device latency, offering a potential solution to bridge the performance
gap between different storage modalities. 

 \par By employing micro-benchmarks we try to discern the inherent trade-offs. This
evaluation is crucial for comprehensively understanding the landscape of storage
solutions and their applicability in real-world scenarios. To assess the
performance of block device setups, we leverage TeraHeap. TeraHeap extends the
capabilities of the JVM and utilizes a second heap stored on a storage device.
This approach allows for a meticulous evaluation of storage solutions, providing
nuanced insights into their performance characteristics. Our evaluation process,
empowered by TeraHeap, furnishes high-level results that offer valuable insights
into the comparative advantages and limitations of each storage solution.

\par The selection among these options requires careful consideration of factors such
as workload requirements and infrastructure configurations. The insights
acquired by this study will play a pivotal role in informing decision-making
processes related to resource allocation and system design, thereby contributing
to the ongoing discourse on efficient data management in contemporary computing
environments.

%\subsection{Key Contribution}