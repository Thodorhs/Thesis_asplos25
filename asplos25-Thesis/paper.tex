%%%%%%%%%%%%%%%%%%%%%%%%%%%%%%%%%%%%%%%%%%%%%%%%%%%%%%%%%%%%%%%%%%%%%%%%%%%%%%%%
% Template for ASPLOS papers.
%
% History:
% 
% ASPLOS originally used jpaper.cls for submission but required acmart.cls for the
% final camera-ready version. To avoid a change in format, starting ASPLOS 2024 Fall 
% cycle, both the submission and the camera-ready versions started using acmart.cls.
%
%%%%%%%%%%%%%%%%%%%%%%%%%%%%%%%%%%%%%%%%%%%%%%%%%%%%%%%%%%%%%%%%%%%%%%%%%%%%%%%%%%

% use the base acmart.cls version 1.92
% use the sigplan proceeding template with the default 10 pt fonts
% nonacm option removes ACM related text in the submission. 
\documentclass[nonacm,sigplan]{acmart}

% enable page numbers
\settopmatter{printfolios=true}

% make references clickable 
\usepackage[]{hyperref}
\usepackage{xcolor}
\usepackage{soul}
\usepackage{listings}
\usepackage{courier}
\usepackage{multirow}
\usepackage{array}
\usepackage{booktabs}
\usepackage{ragged2e} % For text alignment
\usepackage{geometry} % For page layout
\begin{document}

\title{Impact of Exported Fast Block Storage Devices on the Performance of Managed Big Data Frameworks.}

\author{Theodoros Pontzouktzidis} % removed for anonymity

\begin{abstract}
\par Today Big data analytics frameworks running on managed runtimes like Java Virtual Machines (JVM) are extensively used in data centers for analyzing large datasets. Existing research has explored extending the managed heap using fast local storage devices (e.g., NVMe SSDs) and remote memory. Yet, a comprehensive comparison of these techniques is lacking, underscoring the need for detailed studies evaluating their performance, and reliability.
\par In our comparative analysis, we investigate strategies for optimizing block storage access and sizing, focusing on local NVMe SSDs versus remote storage devices and DRAM. We break down and examine the latencies of local and remote storage systems, including SPDK NVMe over Fabrics (NVMe-oF) and Network Block Device (NBD). Using TeraHeap, which extends JVM capabilities by utilizing a second heap on a storage device, we compare its performance with both local and remote NVMe storage options. Our evaluation, conducted using 13 widely-used applications in two real-world big data frameworks, Spark and Giraph, demonstrates that remote NVMe can match the performance of local NVMe. \par This study provides valuable insights into the performance characteristics of various storage solutions, informing decisions on resource allocation and system design for efficient data management in modern computing environments. 
\end{abstract}

\keywords {large analytics datasets, large managed heaps, memory hierarchy, fast storage devices, remote storage, remote memory, Systems Performance}


\maketitle % should come after the abstract
\pagestyle{plain} % should come right after \maketitle

\section{Introduction}
%% Demand for more memory
Big data analytics frameworks running on managed runtimes, such as Java virtual
machines (JVM) are widely deployed in data-centers to perform data analysis over
large amount of datasets. The amount of data increases with high rate but DRAM
capacity in a single server scales slower than data growth. Existing approaches
study the extension of the managed heap over fast storage devices (e.g., NVMe
SSDs) and remote memory.

%Today, cloud infrastructures running data-intensive applications encounter
%numerous challenges in transferring data to-from block storage devices. 
%With the
%escalating demands of big data frameworks, the insufficiency of DRAM in scaling
%to such magnitudes becomes apparent. Additionally, traditional approaches to
%accessing local storage devices struggle to match the pace of data processing,
%thereby becoming an overhead. Consequently, the rapid expansion of data
%necessitates a swift and efficient mechanism for data movement to storage
%devices.

Related work that extend the managed heap beyond local DRAM either uses local storage devices~\cite{XXX} or remote memory~\cite{XXXX}. However, current
research lacks a comprehensive comparison of these techniques. None of the
existing work adequately addresses the pros and cons of each approach or
explains why one might be preferable over the others. This gap highlights a need
for detailed evaluative studies that assess the performance, and reliability of
each heap extension implementation method.

%categorization of how you can have an extended heap approaches like using a extended heap over local storage device or over remote storage device or remote DRAM memory exist...

%However, non of existing work provide a comparison between the techniques on what are the pros and cons and why they are better from one another...

In this comparative analysis, we delve into possible strategies for achieving
faster and more efficient block storage accessing while ensuring an appropriate
size of the block storage. The first aspect of our investigation centers on
comparing local storage devices, exemplified by NVMe SSDs, against remote
storage devices and remote DRAM memory. NVMe SSDs, offer rapid data access and
reliability. However, the efficiency of local storage devices in comparison to
remote options remains a question mark and forms the core of our study. Remote
DRAM memory, despite its remote positioning, may offer lower latency compared to
remote storage devices due to its faster access times. However, its capacity is
typically more constrained than storage devices. It's crucial to note the role
of technological advancements such as SPDK NVMe over Fabrics (NVMe-oF) in
mitigating latency concerns. SPDK NVMe-oF holds promise in reducing device
latency, offering a potential solution to bridge the performance gap between
different storage modalities. 

By employing micro-benchmarks we try to discern the inherent trade-offs. This
evaluation is crucial for comprehensively understanding the landscape of storage
solutions and their applicability in real-world scenarios. To assess the
performance of block device setups, we leverage TeraHeap. TeraHeap extends the
capabilities of the JVM and utilizes a second heap stored on a storage device.
This approach allows for a meticulous evaluation of storage solutions, providing
nuanced insights into their performance characteristics. Our evaluation process,
empowered by TeraHeap, furnishes high-level results that offer valuable insights
into the comparative advantages and limitations of each storage solution.

The selection among these options requires careful consideration of factors such
as workload requirements and infrastructure configurations. The insights
acquired by this study will play a pivotal role in informing decision-making
processes related to resource allocation and system design, thereby contributing
to the ongoing discourse on efficient data management in contemporary computing
environments.

%\subsection{Key Contribution}

\input{back}
\section{Experimental Methodology}
With our evaluation we try to answer: 

\begin{enumerate}[itemsep=1.5pt]
    \item What overheads are present with remote devices in NBD and NVMe-oF.
    \item How does data granularity affect remote and local options (e.g. \SI{512}{B} and \SI{4}{KB}).
    \item Can SPDK NVMe-oF match up against local drive in big data analytics.
\end{enumerate}

%\note{jk: Make the questions more specific. For example, add a question about
%granularity (e.g. 512b and \SI{4}{KB}). So, think carefully what was the evaluation
%about and write specific questions. Do not care if they are too much, we are
%going to reduce/merge them.}
\vspace{1em}

For the evaluation, we use two servers with Intel Xeon E5-2630 32 Cores @ 2.4
GHz. Each server uses 256GB of DDR4 DRAM divided on two NUMA nodes, each with 16
threads. For storage we use Samsung 970 EVO Plus 2 TB PCIe Gen 3.0 x4 NVMe SSD.
Each server is equipped with Mellanox Technologies ConnectX-3 Network controller
MT27500 with fibre ports compliant with the InfiniBand Architecture
Specification ~\cite{infiniband}. The servers run CentOS version 7.9 with Linux kernel version is
5.4.267. Table \ref{tab:serv_specs} shows the server specifications.
\begin{table}[H]
\resizebox{8.5cm}{1cm}{%
\begin{tabular}{cccccc}
\textbf{ID} & \textbf{CPU}                                                                      & \textbf{DRAM}                                        & \textbf{Device}                                                                       & \textbf{NIC}                                                                       & \textbf{Kernel} \\ \hline
Server 1    & \begin{tabular}[c]{@{}c@{}}Intel Xeon E5-2630\\  32 Cores @ 2.4 GHz.\end{tabular} & \begin{tabular}[c]{@{}c@{}}256GB\\ DDR4\end{tabular} & \begin{tabular}[c]{@{}c@{}}Samsung 970 EVO \\ Plus 2 TB PCIe \\ NVMe SSD\end{tabular} & \begin{tabular}[c]{@{}c@{}}Mellanox Technologies \\ ConnectX-3 MT2750\end{tabular} & 5.4.267         \\ \hline
Server 1    & \begin{tabular}[c]{@{}c@{}}Intel Xeon E5-2630\\  32 Cores @ 2.4 GHz.\end{tabular} & \begin{tabular}[c]{@{}c@{}}256GB\\ DDR4\end{tabular} & \begin{tabular}[c]{@{}c@{}}Samsung 970 EVO \\ Plus 2 TB PCIe \\ NVMe SSD\end{tabular} & \begin{tabular}[c]{@{}c@{}}Mellanox Technologies \\ ConnectX-3 MT2750\end{tabular} & 5.4.267        
\end{tabular}
}
\caption{Server specifications table.}
\label{tab:serv_specs}
\end{table}

\paragraph{Flexible I/O (FIO) Tester.}
FIO ~\cite{fio} is a widely used open-source tool in the Linux ecosystem for benchmarking
and testing various I/O (input/output) workloads on storage devices. FIO
provides detailed output reports, including metrics such as throughput, IOPS
(input/output operations per second), latency, and CPU utilization. We configure FIO to use the
libaio engine, random reads, direct I/O, \SI{512}{B} bytes and \SI{4}{KB} block size, 1 I/O
depth and 1 thread to measure the latency of all the device setups shown in Table
\ref{tab:storage_configurations}. 

\begin{table}[h!]
\centering
\begin{tabular}{|>{\centering\arraybackslash}m{3cm}|>{\RaggedRight\arraybackslash}m{5cm}|}
\hline
\textbf{Configuration} & \textbf{Description} \\
\hline
LOC\_n & Local Non-Volatile Memory Express (NVMe) storage drive. \\
\hline
LOC\_r & Storage using a portion of system RAM as a high-speed drive. \\
\hline
NBD\_n & Network Block Device (NBD) configured to use an NVMe drive over the network. \\
\hline
NBD\_r & Network Block Device (NBD) configured to use a RAM-disk over the network. \\
\hline
SPDK\_n & Local NVMe drive managed with Storage Performance Development Kit (SPDK) user-space drivers. \\
\hline
SPDK\_r & Local RAM-disk managed with SPDK user-space drivers. \\
\hline
OF\_n & NVMe over Fabrics (NVMe-oF) using NVMe drives managed with SPDK user-space drivers. \\
\hline
OF\_r & NVMe-oF using RAM-disk managed with SPDK user-space drivers. \\
\hline
\end{tabular}
\caption{Storage device setups.}
\label{tab:storage_configurations}
\end{table}





\paragraph{Netperf.} 
Netperf ~\cite{netperf} is a benchmarking tool used to measure the performance of networking
systems. It's designed to provide a standardized method for measuring networking
performance between two systems. Netperf allows users to test various aspects of
networking performance, such as throughput, latency, and jitter, across
different network protocols like TCP (Transmission Control Protocol) and UDP
(User Datagram Protocol). We will use Netperf to measure end-to-end latency of
TCP (Transmission Control Protocol) To better understand the latency of systems
like Network Block Device (NBD) which uses TCP to export the block device to the
network.

\paragraph{TeraHeap.} TeraHeap ~\cite{teraheap} is a system that eliminates S/D overhead and expensive GC scans for a
large portion of the objects in big data frameworks. TeraHeap enhances the
managed runtime environment, particularly the Java Virtual Machine (JVM). It
introduces a supplementary heap, designed for high-capacity storage, alongside
the primary heap. This secondary heap utilizes fast storage and allows direct
access to objects without the need for serialization or deserialization.
Additionally, TeraHeap minimizes the garbage collection overhead by preventing
the garbage collector from scanning the secondary heap. It takes advantage of
frameworks' capability to designate certain objects for off-heap allocation and
provides them with a hint-based method for relocating these objects to the
secondary heap. We will use TeraHeap as a Real world system. We configure TeraHeap to apply the supplementary heap to local, remote block devices, and remote Ram-disk and compare the performances of each block device system.

\paragraph{ib\_read\_lat.} ib\_read\_lat is a micro-benchmark from the Perftest tool ~\cite{perftest}. specifically designed for measuring InfiniBand (IB) latency. It evaluates the time taken for data to be transferred and received between InfiniBand endpoints. We will use it to measure the latency of transferring data between two Mellanox Technologies ConnectX-3 MT2750 network cards using the IB port.

\paragraph{Workloads and Execution time breakdown.} We employ eight memory-intensive tasks from Spark-Bench ~\cite{spark} and five LDBC Graphalytics suites for Giraph ~\cite{giraph}, generating datasets accordingly. For the remote Ram-disk, we use the five LDBC Graphalytics suites for Giraph and generate a smaller dataset to fit in the limited \SI{100}{G} Ram-disk exported with SPDK NVMe-oF. Each experiment is run five times, and the average end-to-end execution time is recorded. Time breakdown includes 'other' time, S/D plus I/O time, minor GC time, and major GC time. 'Other' time covers mutator threads, potentially including I/O wait. The profiler operates with minimal overhead. We configure TeraHeap to allocate the first heap (H1) on DRAM and the second heap (H2) over a file in Local or Remote block device or Remote Ram-disk via memory-mapped I/O (mmio) we also limmit DRAM and H1 to a fixed size to create pressure and more I/O to H2. Table \ref{tab:workloads} shows the DRAM and H1 size in each workload, in Spark and Giraph, accordingly. 
\begin{table}[H]
\centering
\resizebox{8cm}{3.1cm}{%
\begin{tabular}{|c|l|r|r|}
\hline
\multicolumn{1}{|l|}{\textbf{Frameworks}} & \textbf{Benchmarks} & \multicolumn{1}{c|}{\textbf{TOTAL DRAM (GB)}} & \multicolumn{1}{c|}{\textbf{H1 (GB)}} \\ \hline
\multirow{8}{*}{\rotatebox[origin=c]{90}{\textbf{Spark}}} & Pagerank & 80 & 64 \\ \cline{2-4} 
 & Connected Components & 84 & 68 \\ \cline{2-4} 
 & Linear Regression & 54 & 27 \\ \cline{2-4} 
 & Logistic Regression & 54 & 27 \\ \cline{2-4} 
 & Triangle Counts & 80 & 64 \\ \cline{2-4} 
 & Shortest Path & 58 & 42 \\ \cline{2-4} 
 & SVDPlusPlus & 40 & 24 \\ \cline{2-4} 
 & SVM & 48 & 32 \\ \hline
\multirow{5}{*}{\rotatebox[origin=c]{90}{\textbf{Giraph}}} & PageRank & 85 & 50 \\ \cline{2-4} 
 & CDLP & 85 & 60 \\ \cline{2-4} 
 & WCC & 85 & 60 \\ \cline{2-4} 
 & BFS & 65 & 35 \\ \cline{2-4} 
 & SSSP & 90 & 50 \\ \hline
\multirow{5}{*}{\rotatebox[origin=c]{90}{\shortstack{\textbf{Giraph}\\\textbf{Ram-disk}}}} & PageRank & 16 & 12 \\ \cline{2-4} 
 & CDLP & 16 & 12 \\ \cline{2-4} 
 & WCC & 16 & 12 \\ \cline{2-4} 
 & BFS & 14 & 10 \\ \cline{2-4} 
 & SSSP & 24 & 20 \\ \hline
\end{tabular}%
}
\caption{TeraHeap workload configuration table.}
\label{tab:workloads}
\end{table}


\section{Evaluation}
\subsection{Storage systems latency.}
\note{jk: This section is not well written. It is hard to follow. You should
First we use Flexible I/O (FIO) Tester and \note{jk: micro benchmarks are
offered by FIO you did not write them} create micro benchmarks to understand
how much latency each different storage system adds. \note{start the paragraph
based on what you are doing. We use FIO to measure the latency of ....} We configure FIO to use the
libaio engine, random reads, direct I/O, 512 bytes and 4KB block size, 1 I/O
depth and 1 thread \note{jk: these should be mentioned in methodology section}.

Figure \ref{fig:fio_512} shows the latency with 512 \note{what 512 bytes? kb?
mb?} request size
of each storage system. \note{jk: the next line write it to the caption of the
figure not in the text} "r" stands for ramdisk and "n" stands for NVMe, "OF"
stands for NVMe-oF with SPDK User-Space drivers.

\begin{figure}[H]
  \includegraphics[width=\linewidth]{figures/fio_512.pdf}\\
\caption{Average latency (usec) of local and remote storage systems with
\note{jk: the number should have a single space with the measure 512 B} 512B request size.}
\label{fig:fio_512}
\end{figure}

With local NVMe \note{jk: SSD,} FIO reports an average of 14.16 usec \note{jk:
do not use usec but the symbol of microsecond} latency and
with local ramdisk FIO reports 2.04 usec 6.94× better \note{jk: rewrite the
sentence}. Exported NVMe \note{jk: SSD fix it everywhere} with NBD that uses TCP
\note{jk: avoid to use many parenthesis in the text} latency is 4.22× higher than local with an average of 59.86 usec. By
looking at the Exported ramdisk with NBD (TCP) average latency we can see that
it is very close to NBD NVMe with 55.11 usec only 1.08× better now, meaning that
there is a common overhead. This overhead can either be the NBD I/O path or high
TCP latency. We use Netperf to measure the TCP latency and conduct experiments
with 512 bytes of data in each packet. Netperf reports 37.84 usec average
latency. Adding local NVMe and TCP latency we have 52 usec very close to the NBD
(TCP) latency 59.86 usec. We conclude that the primary overhead is TCP,
accounting for 63.2\% of the average latency in NBD NVMe.

\par Moving on we try to lower the latency of the local I/O path to the device by using userspace drivers (SPDK). We run FIO on the NVMe with SPDK User-Space drivers and we get 8.31 usec average latency 1.7 × better. We also try to lower the network latency with NVMe-oF NVMe with SPDK User-Space drivers and we measure 25.69 usec average latency. Here SPDK ramdisk outperforms everything with 0.33 usec average latency, but the NVMe-oF ramdisk with SPDK User-Space drivers haves 16.66 usec average latency closer to the NVMe-oF NVMe only 1.54× better compared to the local performance where the local ramdisk is 25.18× better. We can also say here that there is a common overhead. This overhead can either be the NVMe I/O path or high Network latency. Previously we measured the NVMe with SPDK User-Space drivers and we got 8.31 usec a logical 32.34\% of the NVMe-oF NVMe with SPDK latency. We used ib\_read\_lat to conduct RDMA Read Latency Test with 512 bytes requests and we measure that the infiniband ports average latency is 2 usec only 7.78\% of the NVMe-oF NVMe with SPDK latency meaning that the high latency can be by the Linux Kernel NVMe-oF Initiator. Unfortunately, SPDK's user-space NVMe-oF initiator doesn't directly create block devices in /dev, SPDK uses its own bdev (block device) layer which will not be compatible for later experiments with TeraHeap. 
\par The NVMe-oF NVMe with SPDK achieving 25.69 microseconds and the NVMe-oF ramdisk with SPDK reaching 16.66 microseconds average latency (nearly as fast as the local NVMe at 14.16 microseconds) are suitable to proceed with our evaluation using Teraheap.
\par Similarly with 4KB block size we get the same overheads as shown in figure \ref{fig:fio_4k}. Here FIO reports more latency for moving the 4KB data locally and/or across network.

\begin{figure}[H]
  \includegraphics[width=\linewidth]{figures/fio_4k.pdf}\\
\caption{Average latency (usec) of local and remote storage systems with 4KB request size.}
\label{fig:fio_4k}
\end{figure}

\subsection{TeraHeap performance with NVMe-oF}
\par This section compares TeraHeap performance of two set ups one with local NVMe device and one with NVMe-oF exported NVMe with SPDK User-Space drivers. Figure \ref{fig:bench_spark} illustrates the performance of TeraHeap with Spark workloads for both setups.
\begin{figure}[H]
  \includegraphics[width=\linewidth]{figures/bench_spark.pdf}\\
\caption{TeraHeap Spark performance. local NVMe device (L) compared to NVMe-oF exported NVMe with SPDK (R).}
\label{fig:bench_spark}
\end{figure}
The two TeraHeap setups local NVMe device (L) and NVMe-oF exported NVMe with SPDK (R) perform similarly with Spark workloads. Spark Workloads read objects from H2 but they don't change them so heavy write operations are missing. We can see that local setup outperforms the remote setup in Pagerank(PR), Connected Components(CC) and Logistic Regression(LgR) making the local setup 0.85\%, 3.55\% and 3.18\% faster accordingly. The big difference is in the SVM workload where the local setup outperforms the remote by 37.30\% \textbf{(why????)}. There are also cases where the remote setup is better. Workloads Linear Regression(LR), Triangle Counts(TR), Shortest Path(SSSP) and SVDPlusPlus(SVD) report that the remote setup is 4.76\%, 2.08\%, 1.37\% and 2.76\% quicker accordingly.
\par Next we run Teraheap with the Giraph Workloads. These workloads read objects from H2 and also can change them. Here we expect heavy write operations that can stress the remote setup. Figure \ref{fig:bench_giraph} illustrates the performance of TeraHeap with Giraph workloads for both setups NVMe device (L) and NVMe-oF exported NVMe with SPDK (R).
\begin{figure}[H]
  \includegraphics[width=\linewidth]{figures/bench_giraph.pdf}\\
\caption{TeraHeap Giraph performance. local NVMe device (L) compared to NVMe-oF exported NVMe with SPDK (R).}
\label{fig:bench_giraph}
\end{figure}
Here, due to the write operations, the remote setup (R) never manages to exceed the performance of the local setup (L). PageRank (PR), CDLP and BFS workloads are the ones that the remote setup struggles the most with local setup being 22.97\% 16.24\% and 13.07\% quicker accordingly. In the setups mentioned before the remote setup is losing performance due to major GC's taking more time to complete. Following up with WCC and SSSP workloads we can see that the local setup is only 1.54\%	and 3.97\% faster again here major GC's are the reason the remote setup is slower.

\subsection{Workload disk statistics}

\par In this section we review the disk statistics of the workload runs. First we examine Spark. figures \ref{fig:spark_r} and \ref{fig:spark_w} show reads and writes accordingly. First we can confirm that reads are more than writes due to the nature of spark benchmarks. Next focusing on the reads we can see that the Gigabytes read for both the local setup (L) and the remote setup (R) are close. An exception here is the SVM workload where the remote setup haves 8.3× more Gigabytes of reads.
\par Next we examine Giraph. figures \ref{fig:giraph_r} and \ref{fig:giraph_w} show reads and writes accordingly. We can confirm that writes are close or more than reads due to Giraph benchmarks changing off-heap objects. focusing on the writes we can see that the write Gigabytes for both the local setup (L) and the remote setup (R) are close. In the PageRank(PR)	and CDLP workloads where the local setup (L) haves the best performance compared to the remote setup (R) (figure \ref{fig:bench_giraph} 22.97\% and 16.24\% faster) we can see more reads (figure \ref{fig:giraph_r}) for the remote setup due to more major GC's.

\begin{figure}[H]
  \includegraphics[width=\linewidth]{figures/spark_r.pdf}\\
\caption{Teraheap Spark workloads reads (GB). local NVMe device (L) compared to NVMe-oF exported NVMe with SPDK (R).}
\label{fig:spark_r}
\end{figure}
\begin{figure}[H]
  \includegraphics[width=\linewidth]{figures/spark_w.pdf}\\
\caption{Teraheap Spark workloads writes (GB). local NVMe device (L) compared to NVMe-oF exported NVMe with SPDK (R).}
\label{fig:spark_w}
\end{figure}
\vspace{10em}
\begin{figure}[H]
  \includegraphics[width=\linewidth]{figures/giraph_r.pdf}\\
\caption{Teraheap Giraph workloads reads (GB). local NVMe device (L) compared to NVMe-oF exported NVMe with SPDK (R).}
\label{fig:giraph_r}
\end{figure}

\begin{figure}[H]
  \includegraphics[width=\linewidth]{figures/giraph_w.pdf}\\
\caption{Teraheap Giraph workloads writes (GB). local NVMe device (L) compared to NVMe-oF exported NVMe with SPDK (R).}
\label{fig:giraph_w}
\end{figure}
%\begin{itemize}
 %   \item start by explaining FIO results and that we will focus on latency
  %  \item show fio results explain the graph (break down loc numbers, say nbd is slow-show netperf tcp latency number, break down spdk numbers(reference to sequential FIO latency number-cache hit))
  %  \item show infiniband numbers (add an ib-read-lat subsection in methodology)
   % \item show spark , giraffe numbers comment on faster or slower times with percentages
    %\item show disk stats numbers and plots to back down heavy I/O workloads
%\end{itemize}

\section{Conclusions}
Big data analytics frameworks on JVMs are widely used in data centers for large dataset analysis. Research has explored extending the managed heap with fast local storage (e.g., NVMe SSDs) and remote memory, but comprehensive performance and reliability comparisons are lacking. In our comparative analysis, we first employed micro-benchmarks to explore the latency and throughput performance of various storage systems, including local NVMe, remote NVMe, and remote memory. We then selectively compared local device systems with NVMe-oF SPDK NVMe and NVMe-oF SPDK Ram-disk using TeraHeap, which extends the managed heap over these devices. We evaluated these options with 13 widely-used applications in two real-world big data frameworks, Spark and Giraph. The workload reports indicate that NVMe-oF and remote options can match the performance of local ones, yielding very similar results. We found out that the NVMe-oF SPDK NVMe struggles to perform well with the Giraph workloads where there are a lot of writes to objects in the second heap (H2) but the spark workloads have very similar results compared to the local device setup. We also concluded that extending the heap over NVMe-oF SPDK Ram-disk can perform almost the same with the local setup. This study provides valuable insights into the performance characteristics of different storage solutions, informing decisions on resource allocation and system design, and contributing to the ongoing discourse on efficient data management in contemporary computing environments.
\input{ack}

\bibliographystyle{plain}
\bibliography{paper}

\end{document}

