\section{Evaluation}
\subsection{Storage systems latency.}
We use FIO to measure how much latency each different storage system adds.
Figure \ref{fig:fio_512} shows the latency with 512 bytes request size
of each storage system.

\begin{figure}[H]
  \includegraphics[width=\linewidth]{figures/fio_512.pdf}\\
\caption{Average latency (usec) of local and remote storage systems with 512 B request size. "r" stands for ramdisk and "n" stands for NVMe, "OF" stands for NVMe-oF with SPDK User-Space drivers.}
\label{fig:fio_512}
\end{figure}

We will analyze the results to determine the best setup for evaluating Teraheap. Starting with local SSD FIO reports an average of 14.16 $\mu$s latency and for local ramdisk, FIO reports 2.04 $\mu$s 6.94$\times$ better. First, we will investigate the remote option NBD (TCP). The exported SSD with NBD average latency is 4.22$\times$ higher than local with an average of 59.86 $\mu$s. Meaning that we get 45.7 $\mu$s more latency with NBD. By looking at the Exported ramdisk with NBD average latency is 1.08$\times$ better than the NBD SSD, meaning that there is a common overhead. This overhead can either be the NBD I/O path or high TCP latency. We use Netperf to measure the TCP latency and conduct experiments with 512 bytes of data in each packet. Netperf reports 37.84 $\mu$s average
latency. Adding local NVMe and TCP latency we have 52 $\mu$s 7.86 $\mu$s difference to the NBD
(TCP) latency 59.86 $\mu$s. We conclude that the primary overhead is TCP, accounting for 63.2\% of the average latency in NBD NVMe.

Next, we investigate lowering the latency of the local I/O path to the device by using userspace drivers (SPDK). With userspace drivers, we bypass the layers of the kernel's I/O stack and eliminate context switches. We run FIO on the NVMe with SPDK and get 8.31 $\mu$s average latency 1.7$\times$ better compared to traditional drivers. We also try to lower
the network latency with NVMe-oF NVMe with SPDK User-Space drivers. With the combination of SPDK’s user-space optimizations, the efficient protocol design of NVMe-oF compared to TCP and RDMA with Infiniband transport layer we manage to measure 25.69 $\mu$s average latency 2.33$\times$ better than SSD with NBD. 

Moving on SPDK ramdisk outperforms everything with 0.33 $\mu$s average latency, but the NVMe-oF ramdisk with SPDK User-Space drivers have 16.66 $\mu$s average latency 9.03 $\mu$s better than the NVMe-oF NVMe and only 1.54$\times$ better compared to the local performance where the local ramdisk is 25.18$\times$ better. There is a common overhead in both NVMe-oF setups which is attributed either to the NVMe I/O path or network latency. Previously we measured the NVMe with SPDK User-Space drivers and got 8.31 $\mu$s a logical 32.34\% of the NVMe-oF NVMe with SPDK latency. We used ib\_read\_lat to conduct RDMA Read Latency Test with 512 bytes requests and we measure that the infiniband ports average latency is 2 $\mu$s only 7.78\% of the NVMe-oF NVMe with SPDK latency meaning that the high latency can be by the Linux Kernel NVMe-oF Initiator which uses traditional kernel I/O calls to the exported /dev device and then the calls are encapsulated and transported to the target via infiniband. Unfortunately, SPDK's user-space NVMe-oF initiator doesn't directly create block devices in /dev, SPDK uses its own bdev (block device) layer which will not be
compatible for later experiments with TeraHeap. 

\begin{figure}[H]
  \includegraphics[width=\linewidth]{figures/fio_4k.pdf}\\
\caption{Average latency (usec) of local and remote storage systems with 4KB request size."r" stands for ramdisk and "n" stands for NVMe, "OF" stands for NVMe-oF with SPDK User-Space drivers.}
\label{fig:fio_4k}
\end{figure}

Similarly, with 4KB block size we get the same overheads as shown in figure
\ref{fig:fio_4k}. Here FIO reports more latency for moving the 4KB data locally
and/or across network. The NVMe-oF NVMe with SPDK achieving 25.69 $\mu$s and the NVMe-oF ramdisk with SPDK reaching 16.66 $\mu$s average latency (nearly as fast as the
local NVMe at 14.16 microseconds) are suitable to proceed with our evaluation
using Teraheap.



\subsection{TeraHeap performance with NVMe-oF}
\par This section compares TeraHeap performance of two setups one with local NVMe device and one with NVMe-oF exported NVMe with SPDK User-Space drivers. Figure \ref{fig:bench_spark} illustrates the performance of TeraHeap with Spark workloads for both setups. 
\begin{figure}[H]
  \includegraphics[width=\linewidth]{figures/bench_spark.pdf}\\
\caption{TeraHeap Spark performance. local NVMe device (L) compared to NVMe-oF exported NVMe with SPDK (R).}
\label{fig:bench_spark}
\end{figure}
The two TeraHeap setups local NVMe device (L) and NVMe-oF exported NVMe with SPDK (R) perform similarly with Spark workloads. We can see that the local setup outperforms the remote setup in Pagerank(PR), Connected Components(CC), and Logistic Regression(LgR) making the local setup 0.85\%, 3.55\% and 3.18\% faster accordingly. The big difference is in the SVM workload where the local setup outperforms the remote by 37.30\% \textbf{(why????)}. There are also cases where the remote setup is better. Workloads Linear Regression(LR), Triangle Counts(TR), Shortest Path(SSSP) and SVDPlusPlus(SVD) report that the remote setup is 4.76\%, 2.08\%, 1.37\% and 2.76\% quicker accordingly. Spark Workloads read objects from H2 but don't change them so heavy write operations are missing.

Next, we run Teraheap with the Giraph Workloads. These workloads read objects from H2 and also change them. Here we expect heavy write operations that can stress the remote setup. Figure \ref{fig:bench_giraph} illustrates the performance of TeraHeap with Giraph workloads for both setups NVMe device (L) and NVMe-oF exported NVMe with SPDK (R).
\begin{figure}[H]
  \includegraphics[width=\linewidth]{figures/bench_giraph.pdf}\\
\caption{TeraHeap Giraph performance. local NVMe device (L) compared to NVMe-oF exported NVMe with SPDK (R).}
\label{fig:bench_giraph}
\end{figure}
Here, due to the write operations, the remote setup (R) never manages to exceed the performance of the local setup (L). PageRank (PR), CDLP and BFS workloads are the ones that the remote setup struggles the most here local setup is 22.97\% 16.24\% and 13.07\% quicker accordingly. In the setups mentioned before the remote setup is losing performance due to major GC's taking more time to complete. Following up with WCC and SSSP workloads we can see that the local setup is only 1.54\%	and 3.97\% faster again here major GC's are the reason the remote setup is slower.

\subsection{Workload disk statistics}

 In this section we review the disk statistics of the workload runs. First we examine Spark. figures \ref{fig:spark_r} and \ref{fig:spark_w} show reads and writes accordingly. First we can confirm that reads are more than writes due to the nature of spark benchmarks. Next focusing on the reads we can see that the Gigabytes read for both the local setup (L) and the remote setup (R) are close. An exception here is the SVM workload where the remote setup haves 8.3$\times$ more Gigabytes of reads.
 
 Next we examine Giraph. figures \ref{fig:giraph_r} and \ref{fig:giraph_w} show reads and writes accordingly. Focusing on the writes we can see that the data in Gigabytes for both the local setup (L) and the remote setup (R) are close. In the PageRank(PR)	and CDLP workloads where the local setup (L) haves the best performance compared to the remote setup (R) (figure \ref{fig:bench_giraph} 22.97\% and 16.24\% faster) we can see more reads (figure \ref{fig:giraph_r}) for the remote setup due to more major GC's.

\begin{figure}[H]
  \includegraphics[width=\linewidth]{figures/spark_r.pdf}\\
\caption{Teraheap Spark workloads reads (GB). local NVMe device (L) compared to NVMe-oF exported NVMe with SPDK (R).}
\label{fig:spark_r}
\end{figure}
\begin{figure}[H]
  \includegraphics[width=\linewidth]{figures/spark_w.pdf}\\
\caption{Teraheap Spark workloads writes (GB). local NVMe device (L) compared to NVMe-oF exported NVMe with SPDK (R).}
\label{fig:spark_w}
\end{figure}
\vspace{10em}
\begin{figure}[H]
  \includegraphics[width=\linewidth]{figures/giraph_r.pdf}\\
\caption{Teraheap Giraph workloads reads (GB). local NVMe device (L) compared to NVMe-oF exported NVMe with SPDK (R).}
\label{fig:giraph_r}
\end{figure}

\begin{figure}[H]
  \includegraphics[width=\linewidth]{figures/giraph_w.pdf}\\
\caption{Teraheap Giraph workloads writes (GB). local NVMe device (L) compared to NVMe-oF exported NVMe with SPDK (R).}
\label{fig:giraph_w}
\end{figure}
